%%%%%%%%%%%%%%%%%%%%%%%%%%%%%%%%%%%%%%%%%
% Friggeri Resume/CV
% XeLaTeX Template
% Version 1.1 (9/2/15)
%
% This template has been downloaded from:
% http://www.LaTeXTemplates.com
%
% Original author:
% Adrien Friggeri (adrien@friggeri.net)
% https://github.com/afriggeri/CV
%
% License:
% CC BY-NC-SA 3.0 (http://creativecommons.org/licenses/by-nc-sa/3.0/)
%
% Important notes:
% This template needs to be compiled with XeLaTeX and the bibliography, if used,
% needs to be compiled with biber rather than bibtex.
%
%%%%%%%%%%%%%%%%%%%%%%%%%%%%%%%%%%%%%%%%%
\documentclass[]{../friggeri-cv} % Add 'print' as an option into the square bracket to remove colors from this template for printing
\usepackage[danish]{babel}
\usepackage{icomma}
%\addbibresource{bibliography.bib} % Specify the bibliography file to include publications

\begin{document}

\header{morten emmanuel }{schiøler}{konsulent@deloitte consulting} % Your name and current job title/field

%----------------------------------------------------------------------------------------
%	SIDEBAR SECTION
%----------------------------------------------------------------------------------------

\begin{aside} % In the aside, each new line forces a line break
\includegraphics[width=\linewidth]{../graphics/pic8x10.jpg}
\section{kontakt}
Nordens Plads 12B,
1. sal, lejlighed nr.47
2000 Frederiksberg
~
+45 30 93 43 16
{\small \href{mschioeler@deloitte.dk}{mschioeler@deloitte.dk}}
\section{sprogkundskaber}
	dansk:  \quad modersmål
	engelsk: \quad flydende
	fransk: \quad begrænset praktisk færdighed
	rumænsk: \quad begrænset praktisk færdighed
%norwegian and swedish
\section{programmering}
{\color{red} $\varheartsuit$}Java \& Git 
Selenium, XPath
HTML, CSS, JavaScript
REST
Ruby on Rails
UNIX/Bash
SQL, SAS, R Statistics
Python, \textsc{MATLAB}
RegExp
%{\color{red} $\varheartsuit$} \LaTeX 
\end{aside}

%----------------------------------------------------------------------------------------
%	EDUCATION SECTION
%----------------------------------------------------------------------------------------
\section{profil}
En nysgærrig, lærenem og dreven konsultent med talent for programmering. Elsker matematik og teknologi, er omhyggelig og ansvarsbevidst, og nyder samarbejde. 
\section{uddannelse}
\begin{entrylist}
%------------------------------------------------
\entry
{2014--2016}
{Kandidat {\normalfont i Transport og Logistik}}
{Danmarks Tekniske Universitet, Kgs. Lyngby}
{Omfattende studier i matematisk modellering, planlægning samt operationsanalyse, CBA og risikoanalyse. Tog kurser som f.eks. Optimering i Java, Heltalsprogrammering, Netværksoptimering og Risk Management. Afsluttet med et 12-tal (gennemsnit 11.25).
}


\entry
{2011--2014}
{Bachelor {\normalfont i Produktion og Konstruktion}}
{Danmarks Tekniske Universitet, Kgs. Lyngby}
{Bredt fagområde, bl.a.\ avanceret matematik og operationsanalyse, statistik og sandsynlighedsregning. Derudover fysik, styrkelære, procesteknik mv.}
%------------------------------------------------
%------------------------------------------------

\end{entrylist}

%----------------------------------------------------------------------------------------
%	WORK EXPERIENCE SECTION
%----------------------------------------------------------------------------------------

\section{erfaring}

\subsection{deloitte consulting}

\begin{entrylist}
%------------------------------------------------
\entry
{2017--2018}
{IT-kompetencecenter for Inddrivelse (ICI)}
{Business System Transformation @ DC}
{\emph{Testautomatiseringsekspert}\\
Udvikling af automatiserede browsertest af inddrivelsesløsningen PSRM. 
\begin{itemize}
\item Etablerede automatiseringsframework til browsertest ("ici-sel") baseret på Selenium i Java, som leverer en væsentlig forbedring i stabilitet, læsbarhed og brugervenlighed i forbindelse med automatisering for PSRM. 
\item Fungerede som teknisk ekspert i ici-sel og Java generelt. 
\item Administrerede Git-repository til automatiserede browsertests. 
\item Modtog præmie for at have faktureret næstflest timer (1374 i pågældende FY) i BST på tværs af WBS-numre på Skatteministeriets konto.
\end{itemize}
}

\end{entrylist}

\subsection{danmarks tekniske universitet}

\begin{entrylist}


\entry
{2016--2017}
{DTU Management}
{Danmarks Tekniske Universitet, Kgs. Lyngby}
{\emph{Videnskabelig assistent}\\
Adskillige akademiske opgaver, involverende modeludvikling, programmering, dataanalyse og grundforskning. Udførte avancereret dataanalyse i SAS og SQL.}

\end{entrylist}


%----------------------------------------------------------------------------------------
%	COMMUNICATION SKILLS SECTION
%----------------------------------------------------------------------------------------
%\section{kvalifikationer}
%\begin{entrylist}
%%------------------------------------------------
%\entry
%{2011--2014}
%{Udvalgte bachelorkurser}
%{DTU}
%{videregående matematik 1 og 2, statistik, sandsynlighedsregning, stokastisk simulation, programmering i Matlab, reguleringsteknik, operationsanalyse}
%\entry
%{2014--}
%{Udvalgte masterkurser}
%{DTU}
%{tidsrækkeanalyse (videregående statistik), multivariat statistik, risikomanagement, GIS og vejtrafikplanlægning,  heltalsprogrammering, netværksoptimering, introduktion til planlægning, planlægningsteori, TEMPOP (transport, økonomi, ledelse, planlægning, organisation og politik), transportmodeller, Avancerede transportmodeller, diskrete valgmodeller, intelligente transportsystemer (ITS),}
%%------------------------------------------------
%%------------------------------------------------
%\end{entrylist}

%----------------------------------------------------------------------------------------
%	AWARDS SECTION
%----------------------------------------------------------------------------------------

%\section{priser}
%\begin{entrylist}
	%------------------------------------------------
	%------------------------------------------------
	%\entry
	%{2018}
	%{Næsthøjest antal timer faktureret}
	%{Deloitte Consulting, BST}
				%{Modtog præmie for at have faktureret næstflest timer (1374 i pågældende FY) på tværs af WBS-numre på Skatteministeriets konto.}
	%--------
	%\entry
	%{2011}
	%{Studentereksamen}
	%{Maribo Gymnasium, STX}
	%{Højeste gennemsnit på årgangen, 11,9$\times$1,03 = 12,3.}
%\end{entrylist}
%----------------------------------------------------------------------------------------
%INTERESTS SECTION
%----------------------------------------------------------------------------------------
\section{interesser}
\textbf{professionelt:} softwareudvikling,  kunstig intelligens, design, matematisk modellering, operationsanalyse,  algoritmer, fysik
\textbf{personligt:} klaver, guitar, kaffe, sprog, spil
%----------------------------------------------------------------------------------------
%	PUBLICATIONS SECTION
%----------------------------------------------------------------------------------------
%
%\section{publications}
%
%\printbibsection{article}{article in peer-reviewed journal} % Print all articles from the bibliography
%
%\printbibsection{book}{books} % Print all books from the bibliography
%
%\begin{refsection} % This is a custom heading for those references marked as "inproceedings" but not containing "keyword=france"
%\nocite{*}
%\printbibliography[sorting=chronological, type=inproceedings, title={international peer-reviewed conferences/proceedings}, notkeyword={france}, heading=subbibliography]
%\end{refsection}
%
%\begin{refsection} % This is a custom heading for those references marked as "inproceedings" and containing "keyword=france"
%\nocite{*}
%\printbibliography[sorting=chronological, type=inproceedings, title={local peer-reviewed conferences/proceedings}, keyword={france}, heading=subbibliography]
%\end{refsection}
%
%\printbibsection{misc}{other publications} % Print all miscellaneous entries from the bibliography
%
%\printbibsection{report}{research reports} % Print all research reports from the bibliography
%
%----------------------------------------------------------------------------------------
%
\end{document}
